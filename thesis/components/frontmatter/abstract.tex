\chapter*{Abstract}

\thispagestyle{front}

The manual analysis of malicious programs is getting even more demanding because of their increase in volume and complexity. Therefore, automated solutions based on a series of static and dynamic heuristics have been introduced to sort through the dense stream of files sent for analysis. The latest approaches propose the usage of artificial intelligence algorithms, which automatically infer models and patterns that can not be easily observed (and later implemented in heuristics) by the analysts.

The present paper has the purpose of showing scenarios in which artificial intelligence brings visible benefits to the analysis of malicious programs. It starts from the presentation of introductory concepts, about malicious applications and artificial intelligence. Further, it details how a platform has been implemented to encompass these two areas, as well as the aspects that have been considered in the development process: pipelining, the division into modules, distributed computing, and configurability, to meet the needs of the end users.

\newpage

\chapter*{Rezumat}

Analiza manuală a programelor malițioase devine din ce în ce mai solicitantă din cauza creșterii în volum și în complexitate a acestora. Prin urmare, s-au introdus soluții automate care se bazează pe o serie de euristici statice și dinamice pentru a tria fluxul dens de fișiere trimise spre a fi analizate. Ultimele abordări propun folosirea de algoritmi de inteligență artificială, ce deduc automat mo\-dele și șabloane ce nu pot fi observate cu ușurință (și ulterior implementate în euristici) de către analiști.

Această lucrare are scopul de a prezenta scenarii în care folosirea inteligenței artificiale aduce beneficii vizibile analizei de programe malițioase. Ea pleacă de la prezentarea unor noțiuni introductive, despre aplicații malițioase și inteligență artificială. Ulterior, detaliază modul în care o platforma a fost implementată pentru a îngloba aceste două domenii, cât și aspectele de care s-a ținut cont în procesul de dezvoltare: \textit{pipelining}, împărțirea pe module, calcul distribuit și configurativitate, pentru a satisface nevoile utilizatorilor finali.

\thispagestyle{front}