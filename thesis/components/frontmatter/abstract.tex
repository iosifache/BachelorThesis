\chapter*{Abstract}

\thispagestyle{front}

Analiza manuală a programelor malițioase a devenit din ce în ce mai solicitantă din cauza creșterii în volum și în complexitate a acestora. Prin urmare, s-au introdus soluții automate care se bazează pe o serie de euristici statice și dinamice pentru a tria fluxul dens de fișiere trimise spre a fi analizate. Ultimele abordări propun folosirea de algoritmi de inteligență artificială, ce deduc automat mo\-dele și șabloane ce nu pot fi observate cu ușurință (și ulterior implementate în euristici) de către analiști.

Această lucrare are scopul de a demonstra că folosirea inteligenței artificiale aduce beneficii vizibile analizei de programe malițioase. Ea pleacă de la prezentarea unor noțiuni introductive, despre aplicații malițioase și inteligență artificială. Ulterior, detaliază întregul ciclu de viață (proiectare, implementare, testare și evaluare) al unei platforme demonstrative, care înglobează aceste două domenii.

\newpage \